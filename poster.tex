% Gemini theme
% https://github.com/anishathalye/gemini
\documentclass[final]{beamer}
\setbeamerfont{caption}{size=\footnotesize}

% ====================
% Packages
% ====================

\RequirePackage{luatex85}
%\usepackage[T1]{fontenc}
% no not with luatex \usepackage[T1]{fontenc}
\usepackage{fontspec}
\usepackage[all]{xy}
\usepackage[size=a0,scale=1.0]{beamerposter}
\usetheme{gemini}
\usecolortheme{mit}
\usepackage{qcircuit}
\usepackage{booktabs}
\usepackage{tikz}
\usetikzlibrary{arrows, fit, positioning}
\usepackage{pgfplots}
\pgfplotsset{compat=1.16}
\usepackage{float}
\usepackage{graphicx}
\usepackage{amsmath, amssymb, physics}
\usepackage{subcaption}
% ====================
% Lengths
% ====================

% If you have N columns, choose \sepwidth and \colwidth such that
% (N+1)*\sepwidth + N*\colwidth = \paperwidth
\newlength{\sepwidth}
\newlength{\colwidth}
\setlength{\sepwidth}{0.024\paperwidth}
\setlength{\colwidth}{0.22\paperwidth}

% ====================
% Tikz styles
% ====================


\newcommand{\separatorcolumn}{\begin{column}{\sepwidth}\end{column}}
\tikzstyle{st}=[lightgray, fill, fill opacity=0.1]
\tikzstyle{odot}=[circle,inner sep=0pt,node contents={$\odot$},scale=1.5]
\tikzstyle{spin}=[-triangle 45, thick]
\tikzstyle{superpeers}=[draw,circle,left,text=black,scale=0.5]
\definecolor{mitred}{cmyk}{0.24, 1.0, 0.78, 0.17}
\definecolor{skyblue}{rgb}{0.53, 0.81, 0.94}


% ====================
% Data
% ====================

\pgfplotstableread[col sep = comma]{./data/real-network-00.csv}\realnetworkzz
\pgfplotstableread[col sep = comma]{./data/real-network-01.csv}\realnetworkzo
\pgfplotstableread[col sep = comma]{./data/real-network-10.csv}\realnetworkoz
\pgfplotstableread[col sep = comma]{./data/real-network-11.csv}\realnetworkoo
\pgfplotstableread[col sep = comma]{./data/sim-network-00.csv}\simnetworkzz
\pgfplotstableread[col sep = comma]{./data/sim-network-01.csv}\simnetworkzo
\pgfplotstableread[col sep = comma]{./data/sim-network-10.csv}\simnetworkoz
\pgfplotstableread[col sep = comma]{./data/sim-network-11.csv}\simnetworkoo
\pgfplotstableread[col sep =comma]{./data/sim-box-cluster-00.csv}\simclusterzz
\pgfplotstableread[col sep = comma]{./data/sim-box-cluster-01.csv}\simclusterzo
\pgfplotstableread[col sep = comma]{./data/sim-box-cluster-10.csv}\simclusteroz
\pgfplotstableread[col sep = comma]{./data/sim-box-cluster-11.csv}\simclusteroo
\pgfplotstableread[col sep = comma]{./data/real-box-cluster-00.csv}\realclusterzz
\pgfplotstableread[col sep = comma]{./data/real-box-cluster-01.csv}\realclusterzo
\pgfplotstableread[col sep = comma]{./data/real-box-cluster-10.csv}\realclusteroz
\pgfplotstableread[col sep = comma]{./data/real-box-cluster-11.csv}\realclusteroo


% ====================
% Title
% ====================

\title{Modeling of Measurement-based Quantum Computing on IBM Q Experience Devices}

\author{Unathi K. Skosana \inst{1} \and Supervisor: Prof. Mark Tame \inst{2}}

\institute[shortinst]{Department of Physics, Stellenbosch University}

% ====================
% Body
% ====================

\begin{document}

    \addtobeamertemplate{headline}{} 
    {
    \begin{tikzpicture}[remember picture,overlay] 
        \node [shift={(-10cm,-4.0cm)}] at (current page.north east)
        {\includegraphics[height=5cm]{logo/logo.pdf}}; 
    \end{tikzpicture} 
    }


    \begin{frame}[t]
        \begin{columns}[t]
            \separatorcolumn

            \begin{column}{\colwidth}
                \begin{block}{One-way quantum computing}
                    %% TODO : References 
                    One-way quantum computing is a one of many frameworks of quantum
                    measurement-based quantum computing (MBQC).

                    In such framework, a quantum computation is not realized through explicit
                    unitary evolution of qubits mediated by unitary quantum gates as in the
                    quantum circuit model.

                    Rather, a computation is realized through a sequential single-qubit
                    measurements on an initial resource state.

                    \begin{figure}[H]
                        \centering
                        \begin{tikzpicture}
                            \draw[-triangle 45, very thick] (1.5, 1)  -- +(0:8.0) node[midway, above] {Information flow};
                            \foreach \i in {0,...,11} {
                                \ifnum \i < 5
                                    \draw[spin] (\i, 0) -- +(90:0.5);
                                \fi
                                \ifnum \i = 5
                                    \draw[spin] (\i, 0) -- +(75:0.5);
                                \fi
                                \ifnum \i = 6
                                    \draw[spin] (\i, 0) -- +(75:0.5);
                                \fi
                                \ifnum \i = 7
                                    \draw[spin] (\i, 0) -- +(125:0.5);
                                \fi
                                \ifnum \i = 8
                                    \draw[spin] (\i, 0) -- +(55:0.5);
                                \fi
                                \ifnum \i > 8
                                    \draw[spin] (\i, 0) -- +(90:0.5);
                                \fi
                            };

                            \foreach \i in {0,...,11} {
                                \ifnum \i = 3
                                    \draw[spin] (\i, -1.25) -- +(90:0.5);
                                \fi
                                \ifnum \i > 3
                                    \node[odot, name=o1, at={(\i,-1)}];
                                \fi
                                \ifnum \i < 3
                                    \node[odot, name=o1, at={(\i,-1)}];
                                \fi
                            };

                            \foreach \i in {0,...,11} {
                                \ifnum  \i = 2 
                                    \node[odot, name=o2, at={(\i, -2)}];
                                \fi
                                \ifnum \i = 3 
                                    \draw[spin] (\i, -2.25) -- +(90:0.5);
                                \fi
                                \ifnum  \i = 4 
                                    \node[odot, name=o2, at={(\i, -2)}];
                                \fi
                                \ifnum  \i < 2 
                                    \draw[spin] (\i, -2.25) -- +(90:0.5);
                                \fi
                                \ifnum \i > 4
                                    \draw[spin] (\i, -2.25) -- +(90:0.5);
                                \fi
                            };

                            \foreach \i in {0,...,11} {
                                \ifnum  \i =  0
                                    \node[odot, name=o3, at={(\i, -3)}];
                                \fi
                                \ifnum \i  > 0
                                    \ifnum \i < 6
                                        \draw[spin] (\i, -3.25) -- +(90:0.5);
                                    \fi
                                \fi
                                \ifnum  \i =  6
                                    \node[odot, name=o3, at={(\i, -3)}];
                                \fi
                                \ifnum \i = 7
                                    \draw[spin] (\i, -3.25) -- +(90:0.5);
                                \fi
                                \ifnum \i > 7
                                    \node[odot, name=o3, at={(\i, -3)}];
                                \fi
                            };

                            \foreach \i in {0,...,11} {
                                \ifnum \i < 7
                                    \node[odot, name=o4, at={(\i, -4)}];
                                \fi
                                \ifnum  \i =  7
                                    \draw[spin] (\i, -4.25) -- +(90:0.5);
                                \fi
                                \ifnum \i > 7
                                    \node[odot, name=o4, at={(\i, -4)}];
                                \fi
                            };

                        \end{tikzpicture}
                        \caption{Propagation of information through an entangled state
                            $\ket{\phi}_{\mathcal{C}}$ via measurements. Tilted
                            arrows represented $x-y$ plane measurements, vertical arrows represent
                            measurements of $\hat{\sigma}_x$ and the circles denote measurement of
                        $\hat{\sigma}_z$}
                        \label{fig:information_propagation_through_a_cluster_state}
                    \end{figure}
                \end{block}

                \begin{block}{Graph states for quantum computation}
                    %%% TODO: References
                    The initial resource states are provided in the form  of highly
                    entangled multi-qubit states called \textbf{graph states}

                    \begin{align}
                        \ket{G} = \displaystyle\prod_{\{k, l\} \in E}
                        \text{CZ}^{kl}\ket{+}^{\otimes |V|}
                        \label{eq:graph_state_preparation}
                    \end{align}

                    The two-qubit gate $\text{CZ}^{kl}$ (controlled
                    $\hat{\sigma}_z$) is applied between pairs of
                    vertices according to the edge set $E$ of the underlying graph
                    $G= (V, E)$, describing the topology of the state.

                    \begin{figure}[H]
                        \centering
                        \begin{subfigure}[b]{0.40\textwidth}
                            \centering
                            \caption{A three-qubit linear graph state}
                            \begin{tikzpicture}[thick,auto]
                                \foreach \place/\name in {{(2,-5)/3}, {(0,-3)/1},
                                {(-2,-5)/2}}
                                    \node[superpeers] (\name) at \place {$\textbf{\name}$};
                                    \path (1) edge (2);
                                    \path (3) edge (1);
                            \end{tikzpicture}
                            \label{fig:triangle_cluster_3q}
                        \end{subfigure}
                        \begin{subfigure}[b]{0.40\textwidth}
                            \centering
                            \caption{A three-qubit triangle graph state}
                            \begin{tikzpicture}[thick,auto]
                                \foreach \place/\name in {{(2,-5)/3}, {(0,-3)/1},
                                {(-2,-5)/2}}
                                    \node[superpeers] (\name) at \place {$\textbf{\name}$};
                                    \path (1) edge (2);
                                    \path (2) edge (3);
                                    \path (1) edge (3);
                            \end{tikzpicture}
                            \label{fig:linear_cluster_3q}
                        \end{subfigure}
                        \begin{subfigure}[b]{0.40\textwidth}
                            \centering
                            \caption{A four-qubit linear graph state}
                            \begin{tikzpicture}[thick,auto]
                                \foreach \place/\name in {{(-4,-1)/1}, {(0,-1)/2},
                                {(0,-5)/3}, {(-4,-5)/4}}
                                    \node[superpeers] (\name) at \place {$\textbf{\name}$};
                                    \path (1) edge (2);
                                    \path (3) edge (4);
                                    \path (2) edge (3);
                            \end{tikzpicture}
                            \label{fig:linear_cluster_4q}
                        \end{subfigure}
                        \begin{subfigure}[b]{0.40\textwidth}
                            \centering
                            \caption{A four-qubit box graph state}
                            \begin{tikzpicture}[thick,auto]
                                \foreach \place/\name in {{(-4,-1)/1}, {(0,-1)/2},
                                {(0,-5)/3}, {(-4,-5)/4}}
                                    \node[superpeers] (\name) at \place {$\textbf{\name}$};
                                    \path (1) edge (2);
                                    \path (3) edge (4);
                                    \path (2) edge (3);
                                    \path (1) edge (4);
                            \end{tikzpicture}
                            \label{fig:box_cluster_4q}
                        \end{subfigure}
                        \caption{A few representives graph states. }
                        \label{fig:graph_states}
                    \end{figure}

                    \heading{Local unitaries}
                    %% TODO : References
                    If two graph state vectors $\ket{G}$ and $\ket{G'}$  and their
                    underlying graphs $G$ and  $G'$ are said to
                    equivalent to another under a local unitary $\hat{U}$ 

                    \begin{align}
                        \ket{G'} = \hat{U}\ket{G}
                        \label{eq:graph_state_unitary_equivalence}
                    \end{align}

                    The graph states $\ket{G}$ and  $\ket{G'}$ are said to belong some
                    equivalence class and within such a class, all graph states are
                    equivalent modulo local unitaries (LU). 


                    \heading{Edge local complementaion and equivalence classes}
                    From an underling graph $G = (V, E)$ of a graph state $\ket{G}$, a new
                    graph $G' = (V, E')$ and an associated graph state $\ket{G'}$ can be
                    realized via the complement of a subgraph induced by the neighborhood of some vertex $k
                    \in V$ of $G$. The two graph states $\ket{G}$ and $\ket{G'}$ are said belong to the same
                    \textbf{LU-equivalence} class.
                \end{block}
            \end{column}

            \separatorcolumn

            \begin{column}{\colwidth}
                \begin{block}{Equivalence classes of graph states}
                    \heading{Unitary for edge complementaion}
                    The mapping $\tau^k : G \mapsto G'$ under edge local complementaion is mediated
                    by a unitary operation $\hat{U}^{k}$

                    \begin{align}
                        \ket{\tau^k(G)} =\hat{U}^{k}(G) \ket{G}
                        \label{eq:edge_local_complementation_mapping}
                    \end{align}

                    where $\hat{U}^{k}$ is of the form

                    \begin{align}
                        \hat{U}^{k}(G) = (-i\hat{\sigma}_x^{k})^{1/2} \displaystyle\prod_{l
                        \in \eta_k} (i \hat{\sigma}_z^{l})^{1/2}
                        \label{eq:edge_local_complementation_unitary}
                    \end{align}

                    \heading{Four-qubit linear graph state $\stackrel{\text{\tiny LU}}{\equiv}$
                    Four-qubit box graph state}

                    Starting from the four-qubit linear graph state as in Fig.
                    ~\ref{fig:linear_cluster_4q}, a sequential application of edge local
                    complementaion on the vertices $3, 2$ and $3$ produces a four-qubit box
                    graph state similar to the one in Fig.~\ref{fig:box_cluster_4q}

                    \begin{figure}[H]
                        \begin{subfigure}[b]{0.25\textwidth}
                            \begin{tikzpicture}[thick,auto]
                                \foreach \place/\name in {{(2,-5)/4}, {(0,-3)/1}, {(-2,-5)/2},
                                {(0, -7)}/3}
                                    \node[superpeers] (\name) at \place {$\textbf{\name}$};
                                    \path (1) edge (2);
                                    \path (2) edge (3);
                                    \path (3) edge (4);
                            \end{tikzpicture}
                        \end{subfigure}%
                        \begin{subfigure}[b]{0.25\textwidth}
                            \begin{tikzpicture}[thick,auto]
                                \foreach \place/\name in {{(2,-5)/4}, {(0,-3)/1}, {(-2,-5)/2},
                                {(0, -7)}/3}
                                    \node[superpeers] (\name) at \place {$\textbf{\name}$};
                                    \path (1) edge (2);
                                    \path (2) edge (3);
                                    \path (3) edge (4);
                                    \path (2) edge (4);
                            \end{tikzpicture}
                        \end{subfigure}%
                        \begin{subfigure}[b]{0.25\textwidth}
                            \begin{tikzpicture}[thick,auto]
                                \foreach \place/\name in {{(2,-5)/4}, {(0,-3)/1}, {(-2,-5)/2},
                                {(0, -7)}/3}
                                    \node[superpeers] (\name) at \place {$\textbf{\name}$};
                                    \path (1) edge (2);
                                    \path (2) edge (3);
                                    \path (2) edge (4);
                                    \path (1) edge (3);
                            \end{tikzpicture}
                        \end{subfigure}%
                        \begin{subfigure}[b]{0.25\textwidth}
                            \begin{tikzpicture}[thick,auto]
                                \foreach \place/\name in {{(2,-5)/4}, {(0,-3)/1}, {(-2,-5)/2},
                                    {(0, -7)}/3}
                                        \node[superpeers] (\name) at \place {$\textbf{\name}$};
                                        \path (1) edge (4);
                                        \path (2) edge (3);
                                        \path (2) edge (4);
                                        \path (1) edge (3);
                            \end{tikzpicture}
                        \end{subfigure}
                        \caption{Sequentially applying the edge local complementaion
                            rule on a four-qubit linear graph state showing that such a
                            state is LU-equivalent to four-qubit box graph state under
                            unitary transformation $\hat{U}^{3}\hat{U}^{2}\hat{U}^{3} =
                            \hat{\sigma}^{1}_{z}\otimes\hat{H}^{2}\otimes\hat{H}^{3}\otimes\hat{\sigma}^{4}_z$}
                            \label{fig:edge_local_complementation_example}
                    \end{figure}

                    Under the vertex-mapping bijection that swaps vertex $1$ and $4$, the
                    graphs in the last step of
                    Fig.~\ref{fig:edge_local_complementation_example} and
                    Fig.~\ref{fig:box_cluster_4q} are isomorphic to one another.


                    % \heading{General algorithm}
                    % For a search space of size $N=2^n$ indexed (by some  $n$-bit integer) elements, we wish to find a specific
                    % item $x^{*}$ identified by some unique index $i^{*} \in \{0, N - 1\}$. A function $f(x)$
                    % implemented by an \textbf{oracle} $O$ defines our a particular search
                    % problem ~\cite{Nielsen2000}.


                    %\begin{align}
                    %    f(x) &= \begin{cases} 
                    %        1 & \text{if $x = x^{*}$} \\
                    %        0 & \text{otherwise}
                    %    \end{cases}
                    %\end{align}

                    % For a classical oracle one would have to consult the oracle at most $N -
                    % 1$ times.

                    %A quantum oracle $\hat{O} : \ket{x} \stackrel{\hat{O}}{\to} (-1)^{f(x)}\ket{x}$ can be constructed such that at most, the oracle is
                    %only consulted $\mathcal{O}(\sqrt{N})$ times.

                    %\heading{Procedure}

                    %\begin{enumerate}
                    %    \item Prepare input register in the state $\ket{0}^{\otimes n}$
                    %    \item Apply Hadamard transform  $H^{\otimes n}$ to put input register in
                    %        $\ket{+}^{\otimes n}$ to prepare $\ket{\psi}$
                    %    \item Apply oracle O
                    %    \item Apply mean inversion operator : $H^{\otimes n}(2\ket{0}\bra{0} -
                    %        I)H^{\otimes n} = 2\ket{\psi}\bra{\psi} - I$
                    %    \item Repeat step 2 and 3 $\mathcal{O}(\sqrt{N})$ times
                    %    \item Measure
                    %\end{enumerate}
                \end{block}
                \begin{block}{Case Study 00 : Grover's quantum search algorithm}
                    %% TODO : References
                    \heading{Two-qubit quantum search on four-qubit box graph state}
                    A two-qubit measurement-based version of Grover's quantum search algorithm can be realized
                    on the four-qubit box graph state.

                    Through Measurements of qubits $1$ and $4$ of Fig.~\ref{fig:box_cluster_4q} in the
                    basis $B_j(\alpha) = \{\ket{+\alpha}_j, \ket{-\alpha}_j\}$
                    where $\ket{\pm \alpha} = \frac{1}{\sqrt{2}}(\ket{0}_j \pm
                    e^{i\alpha}\ket{1}_j) \quad (\alpha \in \mathbb{R})$. Reading out
                    qubits $2$ and $3$ in the basis $B(\pi) =
                    \frac{1}{\sqrt{2}}(\ket{0}_j \mp \ket{1}_j)$.


                    \begin{figure}[H]
                        \raggedright
                        \begin{minipage}[c]{0.40\textwidth}
                            \begin{tikzpicture}[thick,auto] 
                                \draw (-4 , 0) node[label={measure}] {};
                                \draw (0 , 0) node[label={readout}] {};
                                \draw[dashed] (-2.0, 0) -- ++(0, -5.5) {};

                                \foreach \place/\name in {{(0,-1)/2}, {(-3,-1)/1},
                                {(-3,-4)/4}, {(0,-4)/3}}
                                    \node[superpeers] (\name) at \place {$\textbf{\name}$};
                                    \path (1) edge (2);
                                    \path (3) edge (4);
                                    \path (2) edge (3);
                                    \path (1) edge (4);
                            \end{tikzpicture}
                        \end{minipage}%
                        \begin{minipage}[c]{0.40\textwidth}
                            \[
                                \Qcircuit @C=1.0em @R=.6em @!R {
                                    & & \mbox{oracle} & & &  & & \\
                                    \lstick{\ket{+}_2} & \ctrl{1}
                                    & \gate{R_z^{(-\alpha)}}
                                    & \gate{H} & \ctrl{1}
                                    & \gate{\sigma_z}  & \gate{H}
                                    &  \meter & \cw \\
                                    \lstick{\ket{+}_3} & \ctrl{-1}
                                    & \gate{R_z^{(-\beta)}}
                                    & \gate{H} & \ctrl{-1}
                                    & \gate{\sigma_z} & \gate{H}
                                    & \meter & \cw 
                                    \gategroup{2}{2}{3}{3}{.6em}{--}
                                    \gategroup{2}{4}{3}{7}{.6em}{--} \\ 
                                    &  &  & & & \mbox{mean inversion} & & 
                                    }
                            \]
                        \end{minipage}
                    \end{figure}

                    Each of the four cases where the oracle tags the elements
                    $00$, $01$, $10$ and $11$ are realized by the measurement
                    settings $\alpha\beta = $ $\pi\pi$, $\pi0$, $0\pi$ and $00$
                    respectively.

                    \heading{Caveats}
                    The outcome of a measurement in the basis $B_i(\alpha)$ on
                    qubit $i$ $o_i$, which is 0(1) for a measurement of
                    $\ket{+\alpha}(\ket{-\alpha})$.  If the measurement on qubit
                    $1$ and $4$ are anything either than $o_1= 0$ and $o_4= 0$ a
                    Pauli errors are introduced and there is a need to
                    reinterpret the outcomes on qubit $2$ and $3$. In this
                    scenario, a feedforward  of the outcomes must be applied 
                    $\{o_2 \oplus o_4, o_3 \oplus o_ 1\}$.
                \end{block}
            \end{column}

            \separatorcolumn

            \begin{column}{\colwidth}
                \begin{block}{Case study 01 : Performance-improved quantum search}
                    The LU-equivalence and isomorphism of the four-qubit linear and box
                    graph state offers a way to improve the performance of the
                    former implementation.

                    The resultant LU-equivalent four-qubit
                    linear graph state has one less $\text{CZ}$ gate than the
                    four-qubit graph state, which significantly improves upon
                    the performance of the former implementation since two-qubit
                    gates have error rates that are much higher than
                    single-qubit gates.

                    The measurement procedure is the same as in the former
                    implementation, however the feedforward relation has
                    modified to $\{o_2 \oplus o_1, o_3 \oplus o_4 \}$ since
                    qubits $2$ and $3$ have to swapped. Similarly measurement outcomes have
                    to swapped.

                \end{block}
                \begin{block}{Qiskit experiments}
                    To model the algorithm described in case study 00, the
                    quantum circuit in Fig . 

                    \begin{figure}[H]
                        \raggedright
                        \begin{minipage}{0.30\textwidth}
                            \[
                                \Qcircuit @C=1.0em @R=.6em @!R {
                                    \lstick{} & \gate{H} & \ctrl{1}  & \qw       & \qw
                                            & \ctrl{1} & \gate{R_z^{(-\alpha)}} &
                                    \gate{H} & \meter \\
                                    \lstick{} & \gate{H} & \ctrl{-1} & \ctrl{1}  & \qw
                                            & \qw & \gate{Z} & \gate{H} & \meter \\
                                    \lstick{} & \gate{H} & \qw       & \ctrl{-1} & \ctrl{1}
                                            & \qw & \gate{Z} & \gate{H} & \meter \\
                                    \lstick{} & \gate{H} & \qw       & \qw       & \ctrl{-1}
                                            & \ctrl{-3} & \gate{R_z^{(-\beta)}} &
                                    \gate{H} & \meter \inputgroupv{1}{4}{.8em}{.8em}{\ket{0}^{\otimes 4}}
                                }
                            \]
                        \end{minipage}%
                        \begin{minipage}{0.30\textwidth}
                            \[
                                \Qcircuit @C=1.0em @R=.6em @!R {
                                    \lstick{} & \gate{H} & \ctrl{1}  & \qw       & \qw
                                            & \ctrl{1} & \gate{R_z^{(-\alpha)}} &
                                    \gate{H} & \meter \\
                                    \lstick{} & \gate{H} & \ctrl{-1} & \ctrl{1}  & \qw
                                            & \qw & \gate{Z} & \gate{H} & \meter \\
                                    \lstick{} & \gate{H} & \qw       & \ctrl{-1} & \ctrl{1}
                                            & \qw & \gate{Z} & \gate{H} & \meter \\
                                    \lstick{} & \gate{H} & \qw       & \qw       & \ctrl{-1}
                                            & \ctrl{-3} & \gate{R_z^{(-\beta)}} &
                                    \gate{H} & \meter \inputgroupv{1}{4}{.8em}{.8em}{\ket{0}^{\otimes 4}}
                                }
                            \]
                        \end{minipage}

                    \end{figure}

                \end{block}
            \end{column}

            \separatorcolumn

            \begin{column}{\colwidth}
                \begin{block}{Results}
                \end{block}
            \end{column}
        \end{columns}
    \end{frame}

\end{document}
