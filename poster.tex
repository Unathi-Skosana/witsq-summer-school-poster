% Gemini theme
% https://github.com/anishathalye/gemini
\documentclass[final]{beamer}
\setbeamerfont{caption}{size=\footnotesize}

% ====================
% Packages
% ====================

\RequirePackage{luatex85}
%\usepackage[T1]{fontenc}
% no not with luatex \usepackage[T1]{fontenc}
\usepackage{fontspec}
\usepackage[all]{xy}
\usepackage[size=a1]{beamerposter}
\usetheme{gemini}
\usecolortheme{mit}
\usepackage{qcircuit}
\usepackage{booktabs}
\usepackage{tikz}
\usetikzlibrary{arrows, fit, positioning}
\usepackage{pgfplots}
\pgfplotsset{compat=1.16}
\usepackage{float}
\usepackage{graphicx}
\usepackage{amsmath, amssymb, physics}
\usepackage{subcaption}
% ====================
% Lengths
% ====================

% If you have N columns, choose \sepwidth and \colwidth such that
% (N+1)*\sepwidth + N*\colwidth = \paperwidth
\newlength{\sepwidth}
\newlength{\colwidth}
\setlength{\sepwidth}{0.024\paperwidth}
\setlength{\colwidth}{0.22\paperwidth}

% ====================
% Tikz styles
% ====================


\newcommand{\separatorcolumn}{\begin{column}{\sepwidth}\end{column}}
\tikzstyle{st}=[lightgray, fill, fill opacity=0.1]
\tikzstyle{odot}=[circle,inner sep=0pt,node contents={$\odot$}]
\tikzstyle{spin}=[-triangle 45, thick]
\tikzstyle{superpeers}=[draw,circle,left,text=black,scale=0.40]
\definecolor{mitred}{cmyk}{0.24, 1.0, 0.78, 0.17}
\definecolor{skyblue}{rgb}{0.53, 0.81, 0.94}


% ====================
% Data
% ====================



% ====================
% Title
% ====================

\title{Modeling of Measurement-based Quantum Computing on IBM Q Experience Devices}

\author{Unathi K. Skosana \inst{1} \and Supervisor: Prof. Mark Tame \inst{2}}

\institute[shortinst]{Department of Physics, Stellenbosch University}

% ====================
% Body
% ====================

\begin{document}

    \addtobeamertemplate{headline}{} 
    {
    \begin{tikzpicture}[remember picture,overlay, scale=0.75] 
        \node [shift={(-25em,-10em)}] at (current page.north east)
        {\includegraphics[height=15em]{logo/logo.pdf}}; 
    \end{tikzpicture} 
    }


    \begin{frame}[t]
        \begin{columns}[t]
            \separatorcolumn

            \begin{column}{\colwidth}
                \begin{block}{One-way quantum computing}
                    %% TODO : References 
                    One-way quantum computing is one of many frameworks of
                    measurement-based quantum computation (MBQC).

                    In such framework, a quantum computation is not realized
                    through an explicit unitary evolution of qubits mediated by unitary quantum gates as in the
                    quantum circuit model.

                    Rather, a computation is realized through sequential single-qubit
                    measurements on an initial resource
                    state~\cite{PhysRevA.68.022312}.

                    \begin{figure}[H]
                        \centering
                        \begin{tikzpicture}
                            \draw[-triangle 45, very thick] (1.5, 1)  -- +(0:8.0) node[midway, above] {Information flow};
                            \foreach \i in {0,...,11} {
                                \ifnum \i < 5
                                    \draw[spin] (\i, 0) -- +(90:0.5);
                                \fi
                                \ifnum \i = 5
                                    \draw[spin] (\i, 0) -- +(75:0.5);
                                \fi
                                \ifnum \i = 6
                                    \draw[spin] (\i, 0) -- +(75:0.5);
                                \fi
                                \ifnum \i = 7
                                    \draw[spin] (\i, 0) -- +(125:0.5);
                                \fi
                                \ifnum \i = 8
                                    \draw[spin] (\i, 0) -- +(55:0.5);
                                \fi
                                \ifnum \i > 8
                                    \draw[spin] (\i, 0) -- +(90:0.5);
                                \fi
                            };

                            \foreach \i in {0,...,11} {
                                \ifnum \i = 3
                                    \draw[spin] (\i, -1.25) -- +(90:0.5);
                                \fi
                                \ifnum \i > 3
                                    \node[odot, name=o1, at={(\i,-1)}];
                                \fi
                                \ifnum \i < 3
                                    \node[odot, name=o1, at={(\i,-1)}];
                                \fi
                            };

                            \foreach \i in {0,...,11} {
                                \ifnum  \i = 2 
                                    \node[odot, name=o2, at={(\i, -2)}];
                                \fi
                                \ifnum \i = 3 
                                    \draw[spin] (\i, -2.25) -- +(90:0.5);
                                \fi
                                \ifnum  \i = 4 
                                    \node[odot, name=o2, at={(\i, -2)}];
                                \fi
                                \ifnum  \i < 2 
                                    \draw[spin] (\i, -2.25) -- +(90:0.5);
                                \fi
                                \ifnum \i > 4
                                    \draw[spin] (\i, -2.25) -- +(90:0.5);
                                \fi
                            };

                            \foreach \i in {0,...,11} {
                                \ifnum  \i =  0
                                    \node[odot, name=o3, at={(\i, -3)}];
                                \fi
                                \ifnum \i  > 0
                                    \ifnum \i < 6
                                        \draw[spin] (\i, -3.25) -- +(90:0.5);
                                    \fi
                                \fi
                                \ifnum  \i =  6
                                    \node[odot, name=o3, at={(\i, -3)}];
                                \fi
                                \ifnum \i = 7
                                    \draw[spin] (\i, -3.25) -- +(90:0.5);
                                \fi
                                \ifnum \i > 7
                                    \node[odot, name=o3, at={(\i, -3)}];
                                \fi
                            };

                            \foreach \i in {0,...,11} {
                                \ifnum \i < 7
                                    \node[odot, name=o4, at={(\i, -4)}];
                                \fi
                                \ifnum  \i =  7
                                    \draw[spin] (\i, -4.25) -- +(90:0.5);
                                \fi
                                \ifnum \i > 7
                                    \node[odot, name=o4, at={(\i, -4)}];
                                \fi
                            };

                        \end{tikzpicture}
                        \caption{Propagation of information through an entangled state
                            $\ket{\phi}_{\mathcal{C}}$ via measurements. Tilted
                            arrows represented $x-y$ plane measurements, vertical arrows represent
                            measurements of $\hat{\sigma}_x$ and the circles denote measurement of
                        $\hat{\sigma}_z$}
                        \label{fig:information_propagation_through_a_cluster_state}
                    \end{figure}
                \end{block}

                \begin{block}{Graph states for quantum computation}
                    %%% TODO: References
                    The initial resource states are provided in the form of highly
                    entangled multi-qubit states called \textbf{graph states}

                    \begin{align}
                        \ket{G} = \displaystyle\prod_{\{k, l\} \in E}
                        \text{CZ}^{kl}\ket{+}^{\otimes |V|}
                        \label{eq:graph_state_preparation}
                    \end{align}

                    The two-qubit gate $\text{CZ}^{kl}$ (controlled
                    $\hat{\sigma}_z$) is applied between pairs of
                    vertices according to the edge set $E$ of the underlying graph
                    $G= (V, E)$, describing the topology of the state.

                    \begin{figure}[H]
                        \centering
                        \begin{subfigure}[b]{0.40\textwidth}
                            \centering
                            \caption{A three-qubit linear graph state}
                            \begin{tikzpicture}[thick,auto]
                                \foreach \place/\name in {{(2em,-5em)/3}, {(0,-3em)/1},
                                {(-2em,-5em)/2}}
                                    \node[superpeers] (\name) at \place {$\textbf{\name}$};
                                    \path (1) edge (2);
                                    \path (3) edge (1);
                            \end{tikzpicture}
                            \label{fig:triangle_cluster_3q}
                        \end{subfigure}
                        \begin{subfigure}[b]{0.40\textwidth}
                            \centering
                            \caption{A three-qubit triangle graph state}
                            \begin{tikzpicture}[thick,auto]
                                \foreach \place/\name in {{(2em,-5em)/3}, {(0,-3em)/1},
                                {(-2em,-5em)/2}}
                                    \node[superpeers] (\name) at \place {$\textbf{\name}$};
                                    \path (1) edge (2);
                                    \path (2) edge (3);
                                    \path (1) edge (3);
                            \end{tikzpicture}
                            \label{fig:linear_cluster_3q}
                        \end{subfigure}
                        \begin{subfigure}[b]{0.40\textwidth}
                            \centering
                            \caption{A four-qubit linear graph state}
                            \begin{tikzpicture}[thick,auto]
                                \foreach \place/\name in {{(-4em,-1em)/1}, {(0em,-1em)/2},
                                {(0em,-5em)/3}, {(-4em,-5em)/4}}
                                    \node[superpeers] (\name) at \place {$\textbf{\name}$};
                                    \path (1) edge (2);
                                    \path (3) edge (4);
                                    \path (2) edge (3);
                            \end{tikzpicture}
                            \label{fig:linear_cluster_4q}
                        \end{subfigure}
                        \begin{subfigure}[b]{0.40\textwidth}
                            \centering
                            \caption{A four-qubit box graph state}
                            \begin{tikzpicture}[thick,auto]
                                \foreach \place/\name in {{(-4em,-1em)/1}, {(0,-1em)/2},
                                {(0em,-5em)/3}, {(-4em,-5em)/4}}
                                    \node[superpeers] (\name) at \place {$\textbf{\name}$};
                                    \path (1) edge (2);
                                    \path (3) edge (4);
                                    \path (2) edge (3);
                                    \path (1) edge (4);
                            \end{tikzpicture}
                            \label{fig:box_cluster_4q}
                        \end{subfigure}
                        \caption{A few representives graph states. }
                        \label{fig:graph_states}
                    \end{figure}

                    \heading{Local unitaries}
                    %% TODO : References
                    Two graph state vectors $\ket{G}$ and $\ket{G'}$  and their
                    underlying graphs $G$ and  $G'$ are said to
                    equivalent to one another under a local unitary $\hat{U}$ 

                    \begin{align}
                        \ket{G'} = \hat{U}\ket{G}
                        \label{eq:graph_state_unitary_equivalence}
                    \end{align}

                    The graph states $\ket{G}$ and $\ket{G'}$ are said to belong some
                    equivalence class and within such a class, all graph states are
                    equivalent modulo local unitaries (LU). 

                    \heading{Edge local complementaion and equivalence classes}
                    From an underling graph $G = (V, E)$ of a graph state $\ket{G}$, a new
                    graph $G' = (V, E')$ and an associated graph state $\ket{G'}$ can be
                    realized via the complement of a subgraph induced by the neighborhood of some vertex $k
                    \in V$ of $G$. Then the two graph states $\ket{G}$ and $\ket{G'}$ are said belong to the same
                    \textbf{LU-equivalence} class~\cite{PhysRevA.69.062311}.
                \end{block}
            \end{column}

            \separatorcolumn

            \begin{column}{\colwidth}
                \begin{block}{Equivalence classes of graph states}
                    \heading{Unitary for edge complementaion}
                    The mapping $\tau^k : G \mapsto G'$ under edge local complementaion is mediated
                    by a unitary operation
                    $\hat{U}^{k}$~\cite{PhysRevA.69.062311}.

                    \begin{align}
                        \ket{\tau^k(G)} =\hat{U}^{k}(G) \ket{G}
                        \label{eq:edge_local_complementation_mapping}
                    \end{align}

                    where $\hat{U}^{k}$ is of the form

                    \begin{align}
                        \hat{U}^{k}(G) = (-i\hat{\sigma}_x^{k})^{1/2} \displaystyle\prod_{l
                        \in \eta_k} (i \hat{\sigma}_z^{l})^{1/2}
                        \label{eq:edge_local_complementation_unitary}
                    \end{align}

                    \heading{Four-qubit linear graph state $\stackrel{\text{\tiny LU}}{\equiv}$
                    Four-qubit box graph state}

                    Starting from the four-qubit linear graph state as in Fig.
                    ~\ref{fig:linear_cluster_4q}, a sequential application of edge local
                    complementaion on the vertices $3, 2$ and $3$ produces a four-qubit box
                    graph state similar to the one in Fig.~\ref{fig:box_cluster_4q}

                    \begin{figure}[H]
                        \begin{subfigure}[b]{0.25\textwidth}
                            \begin{tikzpicture}[thick,auto]
                                \draw[->] (2.5em, -1.5em) -- ++(1.5em, 0em)
                                    node[below, midway, label={$\hat{U}_3$}] {};
                                \foreach \place/\name in {{(2em,-5em)/4},
                                    {(0em,-3em)/1}, {(-2em,-5em)/2},
                                {(0em, -7em)}/3}
                                    \node[superpeers] (\name) at \place {$\textbf{\name}$};
                                    \path (1) edge (2);
                                    \path (2) edge (3);
                                    \path (3) edge (4);
                            \end{tikzpicture}
                        \end{subfigure}%
                        \begin{subfigure}[b]{0.25\textwidth}
                            \begin{tikzpicture}[thick,auto]
                                \draw[->] (2.5em, -1.5em) -- ++(1.5em,0em)
                                    node[below, midway,
                                    label={$\hat{U}_2$}] {};
                                \foreach \place/\name in {{(2em,-5em)/4},
                                    {(0em,-3em)/1}, {(-2em,-5em)/2},
                                {(0em, -7em)}/3}
                                    \node[superpeers] (\name) at \place {$\textbf{\name}$};
                                    \path (1) edge (2);
                                    \path (2) edge (3);
                                    \path (3) edge (4);
                                    \path (2) edge (4);
                            \end{tikzpicture}
                        \end{subfigure}%
                        \begin{subfigure}[b]{0.25\textwidth}
                            \begin{tikzpicture}[thick,auto]
                                \draw[->] (2.5em, -1.5em) -- ++(1.5em,0em)
                                    node[below,midway, label={$\hat{U}_3$}] {};
                                \foreach \place/\name in {{(2em,-5em)/4},
                                    {(0em,-3em)/1}, {(-2em,-5em)/2},
                                {(0em, -7em)}/3}
                                    \node[superpeers] (\name) at \place {$\textbf{\name}$};
                                    \path (1) edge (2);
                                    \path (2) edge (3);
                                    \path (2) edge (4);
                                    \path (1) edge (3);
                                    \path (1) edge (4);
                            \end{tikzpicture}
                        \end{subfigure}%
                        \begin{subfigure}[b]{0.25\textwidth}
                            \begin{tikzpicture}[thick,auto]
                                \foreach \place/\name in {{(2em,-5em)/4},
                                    {(0em,-3em)/1}, {(-2em,-5em)/2},
                                    {(0em, -7em)}/3}
                                        \node[superpeers] (\name) at \place {$\textbf{\name}$};
                                        \path (1) edge (4);
                                        \path (2) edge (3);
                                        \path (2) edge (4);
                                        \path (1) edge (3);
                            \end{tikzpicture}
                        \end{subfigure}
                        \caption{Sequentially applying the edge local complementaion
                            rule on a four-qubit linear graph state showing that such a
                            state is LU-equivalent to the four-qubit box graph state under
                            unitary transformation $\hat{U}^{3}\hat{U}^{2}\hat{U}^{3} =
                            \hat{\sigma}^{1}_{z}\otimes\hat{H}^{2}\otimes\hat{H}^{3}\otimes\hat{\sigma}^{4}_z$}
                            \label{fig:edge_local_complementation_example}
                    \end{figure}

                    Under the vertex-mapping bijection that swaps vertices $1$ and $4$, the
                    graphs in the last step of
                    Fig.~\ref{fig:edge_local_complementation_example} and
                    Fig.~\ref{fig:box_cluster_4q} are isomorphic to one another.
                \end{block}
                \begin{block}{Case Study 00 : Grover's quantum search algorithm}
                    %% TODO : References
                    \heading{Two-qubit quantum search on four-qubit box graph state}
                    A two-qubit measurement-based version of Grover's quantum search algorithm can be realized
                    on the four-qubit box graph state~\cite{Walther_2005}.

                    Through measurements of qubits $1$ and $4$ of Fig.~\ref{fig:box_cluster_4q} in the
                    basis $B_j(\alpha) = \{\ket{+\alpha}_j, \ket{-\alpha}_j\}$
                    where $\ket{\pm \alpha}_j = \frac{1}{\sqrt{2}}(\ket{0}_j \pm
                    e^{i\alpha}\ket{1}_j) \quad (\alpha \in \mathbb{R})$. Reading out
                    qubits $2$ and $3$ in the basis $B_j(\pi) =
                    \frac{1}{\sqrt{2}}(\ket{0}_j \mp \ket{1}_j)$.

                    \begin{figure}[H]
                        \raggedright
                        \begin{minipage}[c]{0.40\textwidth}
                            \begin{tikzpicture}[thick,auto] 
                                \draw (-4em , 0em) node[label={measure}] {};
                                \draw (0em , 0em) node[label={readout}] {};
                                \draw[dashed] (-2.0em, 0em) -- ++(0em, -5.5em) {};
                                \draw[->] (-4em, -7em) -- ++(4em, 0em) node[midway,
                                    label={time}] {};

                                \foreach \place/\name in {{(0em,-1em)/2}, {(-3em,-1em)/1},
                                {(-3em,-4em)/4}, {(0em,-4em)/3}}
                                    \node[superpeers] (\name) at \place {$\textbf{\name}$};
                                    \path (1) edge (2);
                                    \path (3) edge (4);
                                    \path (2) edge (3);
                                    \path (1) edge (4);
                            \end{tikzpicture}
                        \end{minipage}%
                        \begin{minipage}[c]{0.40\textwidth}
                            \[
                                \Qcircuit @C=1.0em @R=.6em @!R {
                                    & & \mbox{oracle} & & &  & & \\
                                    \lstick{\ket{+}_2} & \ctrl{1}
                                    & \gate{R_z^{(-\alpha)}}
                                    & \gate{H} & \ctrl{1}
                                    & \gate{Z}  & \gate{H}
                                    &  \meter & \cw \\
                                    \lstick{\ket{+}_3} & \ctrl{-1}
                                    & \gate{R_z^{(-\beta)}}
                                    & \gate{H} & \ctrl{-1}
                                    & \gate{Z} & \gate{H}
                                    & \meter & \cw 
                                    \gategroup{2}{2}{3}{3}{.6em}{--}
                                    \gategroup{2}{4}{3}{7}{.6em}{--} \\ 
                                    &  &  & & & \mbox{mean inversion} & & 
                                    }
                            \]
                        \end{minipage}
                    \end{figure}

                    Each of the four cases, where the oracle tags the elements
                    $00$, $01$, $10$ and $11$ are realized by the measurement
                    settings $\alpha\beta = $ $\pi\pi$, $\pi0$, $0\pi$ and $00$
                    respectively.

                    \heading{Caveats}
                    The outcome of a measurement in the basis $B_i(\alpha)$ on
                    qubit $i$ is $o_i$, which is 0(1) for a measurement of
                    $\ket{+\alpha}(\ket{-\alpha})$.

                    If the measurement on qubits $1$ and $4$ are anything either
                    than $o_1= 0$ and $o_4= 0$, Pauli errors are introduced and
                    there is a need to reinterpret the outcomes on qubits $2$ and $3$. In such a scenario, a
                    feedforward of the outcomes must be applied $\{o_2, o_3\}
                    \mapsto \{o_2 \oplus o_4, o_3 \oplus o_ 1\}$.
                \end{block}
            \end{column}

            \separatorcolumn

            \begin{column}{\colwidth}
                \begin{block}{Case study 01 : Improved quantum search}
                    The LU-equivalence and isomorphism of the four-qubit linear and box
                    graph state offers a way to improve the performance of the
                    former implementation.

                    The resultant LU-equivalent four-qubit
                    linear graph state has one less $\text{CZ}$ gate than the
                    four-qubit graph state, which should have an improvement  upon
                    the performance of the former implementation since generally
                    NISQ two-qubit gates are much more error-prone than single-qubit gates.

                    The measurement procedure is the same as in the former
                    implementation, however the feedforward relation has
                    modified to $\{o_2, o_3\} \mapsto \{o_2 \oplus o_1, o_3 \oplus o_4 \}$ since
                    qubits $1$ and $4$ have to swapped.
                \end{block}
                \begin{block}{Qiskit experiments}
                    To model the algorithms described in the two case studies, the
                    quantum circuits in Fig.~\ref{fig:qiskit_box_graph_state}
                    and Fig.~\ref{fig:qiskit_linear_graph_state} were executed on the
                    IBM Q 5 vigo machine, with trials of $8100$ shots.

                    Since the IBM Q 5 vigo machine doesn't have the necessary
                    topology to realize a four-qubit box graph state, the
                    mapping of logical qubits to physical qubits was chosen as to minimize the
                    number of swap gates in the resulting transpiled circuit.

                    To further limit the number of gates, the feedforward
                    operations which would be CNOT gates on the
                    relevant qubits were done post-experiment. Similarly, the swap of
                    qubits $2$ and $3$ in the algorithm in case study $01$ was
                    also done post-experiment.

                    The four-qubit linear graph state passed an multi-partite
                    stabilizer-based entanglement witness
                    test~\cite{PhysRevA.72.022340}, giving an expectation value of
                    $\expval{\mathcal{W}} = -0.45 \pm 0.261$ while the four-qubit
                    box graph state narrowly failed the entanglement witness
                    test, giving $\expval{\mathcal{W}} = 0.003 \pm 0.260$.

                    \begin{figure}[H]
                        \centering
                        \[
                            \Qcircuit @C=2.0em @R=.6em @!R {
                                & &  \mbox{state prep} & & & & & \mbox{$B_1(\alpha),
                        B_{2/3}(\pi) \text{ and } B_4(\beta)$} & \\
                                \lstick{} & \gate{H} & \ctrl{1}  & \qw       & \qw
                                        & \ctrl{1} & \gate{R_z^{(-\alpha)}}
                                        & \gate{H} & \meter & \cw \\
                                \lstick{} & \gate{H} & \ctrl{-1} & \ctrl{1}  & \qw
                                        & \qw & \gate{Z} & \gate{H} & \meter
                                        & \cw \\
                                \lstick{} & \gate{H} & \qw       & \ctrl{-1} & \ctrl{1}
                                        & \qw & \gate{Z} & \gate{H} & \meter
                                        & \cw \\
                                \lstick{} & \gate{H} & \qw       & \qw       & \ctrl{-1}
                                        & \ctrl{-3} & \gate{R_z^{(-\beta)}}
                                        & \gate{H} & \meter & \cw
                                        \gategroup{2}{2}{5}{6}{.7em}{--} 
                                        \gategroup{2}{7}{5}{9}{.7em}{--} 
                                        \inputgroupv{2}{5}{.8em}{.8em}{\ket{0}^{\otimes 4}}
                            }
                        \]
                        \caption{Measurement-based two-qubit Grover's algorithm
                        on a four-qubit box graph state}
                        \label{fig:qiskit_box_graph_state}
                    \end{figure}
                    \begin{figure}
                        \centering
                        \[
                            \Qcircuit @C=2.0em @R=.6em @!R {
                                & &  \mbox{state prep} & & &
                            \mbox{$\hat{U}^3\hat{U}^2\hat{U}^3$} & & 
                                \mbox{$B_1(\alpha), B_{2/3}(\pi) \text{ and }
                                    B_{4}(\beta)$}  & \\
                                \lstick{} & \gate{H} & \ctrl{1}  & \qw       & \qw
                                          & \gate{Z} & \gate{R_z^{(-\alpha)}} &
                                \gate{H} & \meter  & \cw \\
                                \lstick{} & \gate{H} & \ctrl{-1} & \ctrl{1}  & \qw
                                            & \gate{H} & \gate{Z} & \gate{H} & \meter
                                        & \cw \\
                                \lstick{} & \gate{H} & \qw       & \ctrl{-1} & \ctrl{1}
                                            & \gate{H} & \gate{Z} & \gate{H} & \meter
                                        & \cw \\
                                \lstick{} & \gate{H} & \qw       & \qw       & \ctrl{-1}
                                          & \gate{Z} & \gate{R_z^{(-\beta)}} 
                                        & \gate{H} & \meter & \cw
                                        \gategroup{2}{2}{5}{5}{.7em}{--} 
                                        \gategroup{2}{6}{5}{6}{.7em}{--} 
                                        \gategroup{2}{7}{5}{9}{.7em}{--} 
                                        \inputgroupv{2}{5}{.8em}{.8em}{\ket{0}^{\otimes 4}}
                            }
                        \]
                        \caption{Measurement-based two-qubit Grover's algorithm
                        on a four-qubit linear graph state}
                        \label{fig:qiskit_linear_graph_state}
                    \end{figure}
                \end{block}
            \end{column}

            \separatorcolumn

            \begin{column}{\colwidth}
                \begin{block}{Results}
                    \heading{Case study 00}
                    The results below were obtained for the algorithm described
                    in case study $00$, showing the feedforward and no feedforward
                    results. In all cases,the success probabilities are
                    approximately 75\%.
                    \pgfplotstableread{data/vigo_gcl.dat}{\data}           % Read
                    \pgfplotstablegetrowsof{data/vigo_gcl.dat}
                    \pgfmathtruncatemacro{\rows}{\pgfplotsretval-1}       % Put the number of row minus one in \rows, 
                    \pgfplotstablegetelem{0}{[index] 0}\of{\data} 
                            \let\maxX\pgfplotsretval 
                    \pgfplotstablegetelem{0}{[index] 1}\of{\data} 
                            \let\maxY\pgfplotsretval 
                    \pgfplotstablegetelem{0}{[index] 2}\of{\data} 
                            \let\maxZ\pgfplotsretval 
                    \pgfmathsetmacro{\Zscale}{4/\maxZ} % contain the values of z between 0 and 4 cm 
                    \begin{figure}[H]
                        \centering
                        \begin{tikzpicture}[x={(0.86em,-0.5em)},y={(0.866em,0.5em)},z={(0em,\Zscale
                            em)}, scale=1.5]

                        % Defining hsb color to have a color scale
                        \colorlet{redhsb}[hsb]{red}     %
                        \colorlet{bluehsb}[hsb]{blue}   % 
                        \pgfmathsetmacro{\w}{0.35/2} % half the width of the bars

                        % Drawing the system of axes
                        \draw[->] (0,-0.5,0) -- (12,-0.5,0) node [black,left] {};
                        \draw[->] (0,-0.5,0) -- (0,6,0) node [black,left] {};
                        \draw[->] (0,-0.5,0) -- (0,-0.5,1.2) node [black,left] {};

                        \draw (1.5, -1.80, 0) node[label={00}] {};
                        \draw (4.75, -1.80, 0) node[label={01}] {};
                        \draw (7.75, -1.80, 0) node[label={10}] {};
                        \draw (10.75, -1.80, 0) node[label={11}] {};


                        \coordinate (aux3) at (2, -2.5);
                        \coordinate (aux4) at (9, -2.5);

                        \draw[draw=none] (aux3)  -- node[fill=white,sloped] {Tagged element} (aux4);


                        \draw[-] (0,-0.5,0.0) node [black,left] {0.00};
                        \draw[-] (0,-0.5,0.25) node [black,left] {0.25};
                        \draw[-] (0,-0.5,0.50) node [black,left] {0.50};
                        \draw[-] (0,-0.5,0.75) node [black,left] {0.75};
                        \draw[-] (0,-0.5,1.0) node [black,left] {1.00};

                        \coordinate (aux5) at (12,0);
                        \coordinate (aux6) at (14,0);

                        \coordinate (aux7) at (12,4);
                        \coordinate (aux8) at (14,4);


                        \draw[<-] (aux5)  -- node[sloped, above] {FF} (aux6);
                        \draw[<-] (aux7)  -- node[sloped, above] {NO FF} (aux8);



                        \foreach \p in {1,...,\rows}{
                                \pgfplotstablegetelem{\p}{[index] 0}\of{\data}    % The order in which the bars are drawn is determined by
                                \let\x\pgfplotsretval                                   %    the order of the lines in the data file.
                                \pgfplotstablegetelem{\p}{[index] 1}\of{\data}    % And as the drawings just pile up, the last one just goes
                                \let\y\pgfplotsretval                                   %    on top of the previous drawings.
                                \pgfplotstablegetelem{\p}{[index] 2}\of{\data}    % The order here works with chosen view angle, if you
                                \let\z\pgfplotsretval                                   %    change the angle, you might have to change it.


                                \pgfmathtruncatemacro{\teinte}{100 - Mod(2*\x, 6)*100/4}
                                \colorlet{col}[rgb]{bluehsb!\teinte!redhsb}   
                                % Visible faces from original view
                                    \fill[col] (\x+\w,\y+\w,\z) -- (\x+\w,\y-\w,\z) -- (\x+\w,\y-\w,0) -- (\x+\w,\y+\w,0) -- (\x+\w,\y+\w,\z);
                                        \draw[black](\x+\w,\y+\w,\z) -- (\x+\w,\y-\w,\z) -- (\x+\w,\y-\w,0) -- (\x+\w,\y+\w,0) -- (\x+\w,\y+\w,\z);
                                    \fill[col!60!gray] (\x+\w,\y-\w,\z) -- (\x-\w,\y-\w,\z) -- (\x-\w,\y-\w,0) -- (\x+\w,\y-\w,0) -- (\x+\w,\y-\w,\z);
                                        \draw[black](\x+\w,\y-\w,\z) -- (\x-\w,\y-\w,\z) -- (\x-\w,\y-\w,0) -- (\x+\w,\y-\w,0) -- (\x+\w,\y-\w,\z);
                                % Top face
                                \fill[top color=col!40!gray, bottom color=col!80!gray] (\x-\w,\y-\w,\z) -- (\x-\w,\y+\w,\z) -- (\x+\w,\y+\w,\z) -- (\x+\w,\y-\w,\z) -- (\x-\w,\y-\w,\z) ;
                                    \draw[black] (\x-\w,\y-\w,\z) -- (\x-\w,\y+\w,\z) -- (\x+\w,\y+\w,\z) -- (\x+\w,\y-\w,\z) -- (\x-\w,\y-\w,\z);
                        }

                        \end{tikzpicture}
                        \caption{The measured outputs of Grover's algorithm on ibmq\_vigo. The data labeled as
                        'NO FF' show the outputs in those cases the outputs $\{o_1, o_4\}$ were both
                        zero. The data labeled 'FF' show the outputs to which the feedfoward relation
                        was applied}
                    \end{figure}

                    \heading{Case study 01}
                    The results below were those obtained for the algorithm
                    described in case study $01$. Success probabilities are all
                    well over 80\%, making this computation almost as good as the computation in
                    the quantum circuit model computation albeit the greater
                    number of gates.
                    \pgfplotstableread{data/vigo_glu.dat}{\data}           % Read
                    \pgfplotstablegetrowsof{data/vigo_glu.dat}
                    \pgfmathtruncatemacro{\rows}{\pgfplotsretval-1}       % Put the number of row minus one in \rows, 
                    \pgfplotstablegetelem{0}{[index] 0}\of{\data} 
                            \let\maxX\pgfplotsretval 
                    \pgfplotstablegetelem{0}{[index] 1}\of{\data} 
                            \let\maxY\pgfplotsretval 
                    \pgfplotstablegetelem{0}{[index] 2}\of{\data} 
                            \let\maxZ\pgfplotsretval 
                    \pgfmathsetmacro{\Zscale}{4/\maxZ} % contain the values of z between 0 and 4 cm 
                    \begin{figure}[H]
                        \centering
                        \begin{tikzpicture}[x={(0.866em,-0.5em)},y={(0.866em,0.5em)},z={(0em,\Zscale
                            em)}, scale=1.5]

                        % Defining hsb color to have a color scale
                        \colorlet{redhsb}[hsb]{red}     %
                        \colorlet{bluehsb}[hsb]{blue}   % 
                        \pgfmathsetmacro{\w}{0.35/2} % half the width of the bars

                        % Drawing the system of axes
                        \draw[->] (0,-0.5,0) -- (12,-0.5,0) node [black,left] {};
                        \draw[->] (0,-0.5,0) -- (0,6,0) node [black,left] {};
                        \draw[->] (0,-0.5,0) -- (0,-0.5,1.2) node [black,left] {};

                        \draw (1.5, -1.8, 0) node[label={00}] {};
                        \draw (4.75, -1.8, 0) node[label={01}] {};
                        \draw (7.75, -1.8, 0) node[label={10}] {};
                        \draw (10.75, -1.8, 0) node[label={11}] {};


                        \coordinate (aux3) at (2, -2.5);
                        \coordinate (aux4) at (9, -2.5);

                        \draw[draw=none] (aux3)  -- node[fill=white,sloped] {Tagged element} (aux4);


                        \draw[-] (0,-0.5,0.0) node [black,left] {0.00};
                        \draw[-] (0,-0.5,0.25) node [black,left] {0.25};
                        \draw[-] (0,-0.5,0.50) node [black,left] {0.50};
                        \draw[-] (0,-0.5,0.75) node [black,left] {0.75};
                        \draw[-] (0,-0.5,1.0) node [black,left] {1.00};

                        \coordinate (aux5) at (12,0);
                        \coordinate (aux6) at (14,0);

                        \coordinate (aux7) at (12,4);
                        \coordinate (aux8) at (14,4);


                        \draw[<-] (aux5)  -- node[sloped, above] {FF} (aux6);
                        \draw[<-] (aux7)  -- node[sloped, above] {NO FF} (aux8);



                        \foreach \p in {1,...,\rows}{
                                \pgfplotstablegetelem{\p}{[index] 0}\of{\data}    % The order in which the bars are drawn is determined by
                                \let\x\pgfplotsretval                                   %    the order of the lines in the data file.
                                \pgfplotstablegetelem{\p}{[index] 1}\of{\data}    % And as the drawings just pile up, the last one just goes
                                \let\y\pgfplotsretval                                   %    on top of the previous drawings.
                                \pgfplotstablegetelem{\p}{[index] 2}\of{\data}    % The order here works with chosen view angle, if you
                                \let\z\pgfplotsretval                                   %    change the angle, you might have to change it.


                                \pgfmathtruncatemacro{\teinte}{100 - Mod(2*\x, 6)*100/4}
                                \colorlet{col}[rgb]{bluehsb!\teinte!redhsb}   
                                % Visible faces from original view
                                    \fill[col] (\x+\w,\y+\w,\z) -- (\x+\w,\y-\w,\z) -- (\x+\w,\y-\w,0) -- (\x+\w,\y+\w,0) -- (\x+\w,\y+\w,\z);
                                        \draw[black](\x+\w,\y+\w,\z) -- (\x+\w,\y-\w,\z) -- (\x+\w,\y-\w,0) -- (\x+\w,\y+\w,0) -- (\x+\w,\y+\w,\z);
                                    \fill[col!60!gray] (\x+\w,\y-\w,\z) -- (\x-\w,\y-\w,\z) -- (\x-\w,\y-\w,0) -- (\x+\w,\y-\w,0) -- (\x+\w,\y-\w,\z);
                                        \draw[black](\x+\w,\y-\w,\z) -- (\x-\w,\y-\w,\z) -- (\x-\w,\y-\w,0) -- (\x+\w,\y-\w,0) -- (\x+\w,\y-\w,\z);
                                % Top face
                                \fill[top color=col!40!gray, bottom color=col!80!gray] (\x-\w,\y-\w,\z) -- (\x-\w,\y+\w,\z) -- (\x+\w,\y+\w,\z) -- (\x+\w,\y-\w,\z) -- (\x-\w,\y-\w,\z) ;
                                    \draw[black] (\x-\w,\y-\w,\z) -- (\x-\w,\y+\w,\z) -- (\x+\w,\y+\w,\z) -- (\x+\w,\y-\w,\z) -- (\x-\w,\y-\w,\z);
                        }

                        \end{tikzpicture}
                        \caption{The measured outputs of Grover's algorithm on
                            ibmq\_vigo. The data labeled as
                        'NO FF' show the outputs in those cases the outputs $\{o_1, o_4\}$ were both zero. The data labeled 'FF' show the outputs to which the feedforward relation
                        was applied}
                    \end{figure}
                \end{block}
                \begin{block}{References}
                    \nocite{*}
                    \footnotesize{\bibliographystyle{plain}\bibliography{poster}}
                \end{block}
            \end{column}

            \separatorcolumn
        \end{columns}
    \end{frame}

\end{document}
